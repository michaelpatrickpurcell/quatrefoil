\documentclass[a6paper, parskip=half, DIV=16, 10pt]{scrartcl}
\usepackage{quatrefoil}
\usepackage{multicol}
\setlength\columnsep{3em}
\usepackage{enumitem}
\usepackage{caption}
\usepackage{scrlayer-scrpage} % Manage headers and footers in Koma-Script document classes
\setlength{\footskip}{1cm}

\usepackage[hidelinks]{hyperref} % Add hyperlinks to the pdf file. This should usually be the last package loaded before \begin{document}

\makeatletter
\newcommand{\version}[1]{\newcommand{\@version}{#1}}
\makeatother

% Set header
\clearpairofpagestyles
\makeatletter
\cfoot*{\normalshape Version \@version}
\makeatother

% Minimize unwanted hyphenation
\tolerance=1
\emergencystretch=\maxdimen
\hyphenpenalty=10000
\hbadness=10000

\setkomafont{section}{\setmainfont{Quattrocento}\Large\bfseries}
\setkomafont{descriptionlabel}{\setmainfont{Quattrocento}\normalsize\bfseries}

\RedeclareSectionCommand[
  runin=false,
  afterindent=false,
  beforeskip=0pt,%0.25\baselineskip,
  afterskip=0ex,%,0.125\baselineskip,
]{section}

\version{1.0}
\begin{document}
{%
\begin{center}
\setmainfont[Scale=2.0]{Quattrocento}
\Huge
Quatrefoil
\vfill
\setmainfont[Scale=2.0]{Quattrocento}
\Large
Rule Book
\end{center}
\begin{center}
\begin{tikzpicture}[scale=4.5, transform shape]
	\node[circle, fill=Green, draw, minimum size=15] () at (-4,-0.035) {};
	\node[circle, fill=Red, draw, minimum size=15] () at (-4,-0.7) {};
	\node[circle, fill=Purple, draw, minimum size=15] () at (-4.665, -0.7) {};
	\node[circle, fill=Blue, draw, minimum size=15] () at (-4.665,-0.035) {};

	\node[rotate=0] () at (-4, -0.09) {{\setmainfont{Quattrocento}\Huge Q}};
	\node[rotate=-90] () at (-4.055, -0.7) {{\setmainfont{Quattrocento}\Huge Q}};
	\node[rotate=180] () at (-4.665, -0.645) {{\setmainfont{Quattrocento}\Huge Q}};
	\node[rotate=90] () at (-4.610, -0.035) {{\setmainfont{Quattrocento}\Huge Q}};
\end{tikzpicture}
\end{center}%
\begin{center}
\setmainfont[Scale=2.0]{Parisienne}
\Large
Michael Purcell
\end{center}
}%
\newpage
%\setcounter{page}{1}
%\clearpairofpagestyles
%\makeatletter
%\cfoot*{\normalshape\arabic{page}}
%\makeatother
\setmainfont{Quattrocento Sans}%
\raggedright%
\section*{Overview}
Quatrefoil is a cooperative puzzle game for four players that can be played in 5-10 minutes. It is intended for players who are at least eight years old.

During the game, you will work together to solve a puzzle. Communication is limited so you will need to be creative and attentive to coordinate your efforts.

You can end the game whenever you think that you have solved the puzzle. If you are right, everyone wins.  If not, everyone loses.

\section*{Components}
\begin{description}[leftmargin=0pt, labelsep=\widthof{\ }]
	\item[Tokens (5) \textendash] One circular token in five colours.%: blue, green, purple, red, yellow.
	\begin{center}
	\begin{tikzpicture}[scale=0.1, transform shape, x=1in, y=1in]
		\pic () at (0,0) {token2={Blue}};
		\pic () at (3,0) {token2={Green}};
		\pic () at (6,0) {token2={Purple}};
		\pic () at (9,0) {token2={Red}};
		\pic () at (12,0) {token2={Yellow}};
	\end{tikzpicture}
	\end{center}
	\item[Colour Cards (5) \textendash] One for each colour.
	\begin{center}
%	\begin{tikzpicture}[scale=0.1, transform shape, x=1in, y=1in]
%		\pic at (0,0) {colorcardfrontdisplay2={Blue}};
%		\pic at (3,0) {colorcardfrontdisplay2={Green}};
%		\pic at (6,0) {colorcardfrontdisplay2={Purple}};
%		\pic at (9,0) {colorcardfrontdisplay2={Red}};
%		\pic at (12,0) {colorcardfrontdisplay2={Yellow}};
%	\end{tikzpicture}
	\begin{tikzpicture}[scale=0.1, transform shape, x=1in, y=1in]
		\pic[rotate=-60, transform shape] at (0:1.5) {colorcardfrontdisplay2={Blue}};
		\pic[rotate=-30, transform shape] at (45:1.5) {colorcardfrontdisplay2={Green}};
		\pic[rotate=0, transform shape] at (90:1.5) {colorcardfrontdisplay2={Purple}};
		\pic[rotate=30, transform shape] at (135:1.5) {colorcardfrontdisplay2={Red}};
		\pic[rotate=60, transform shape] at (180:1.5) {colorcardfrontdisplay2={Yellow}};
	\end{tikzpicture}
	\qquad\begin{tikzpicture}[scale=0.1, transform shape, x=1in, y=1in]
		\pic[rotate=-60, transform shape] at (0:1.5) {cardbackdisplay2={Red}};
		\pic[rotate=-30, transform shape] at (45:1.5) {cardbackdisplay2={Red}};
		\pic[rotate=0, transform shape] at (90:1.5) {cardbackdisplay2={Red}};
		\pic[rotate=30, transform shape] at (135:1.5) {cardbackdisplay2={Red}};
		\pic[rotate=60, transform shape] at (180:1.5) {cardbackdisplay2={Red}};
	\end{tikzpicture}

\end{center}
	\item[Position Cards (4) \textendash] One for each of the ordinal directions: northwest, northeast, southwest, southeast.%The backs of the position cards are blue.
	\begin{center}	
%	\begin{tikzpicture}[scale=0.1, transform shape, x=1in, y=1in]
%		\pic at (0,0) {positioncardfrontdisplay2={Northwest}};
%		\pic at (3,0) {positioncardfrontdisplay2={Northeast}};
%		\pic at (6,0) {positioncardfrontdisplay2={Southwest}};
%		\pic at (9,0) {positioncardfrontdisplay2={Southeast}};
%	\end{tikzpicture}
	\begin{tikzpicture}[scale=0.1, transform shape, x=1in, y=1in]
		\pic[rotate=-45, transform shape] at (0:1) {positioncardfrontdisplay2={Northwest}};
		\pic[rotate=-15, transform shape] at (60:1) {positioncardfrontdisplay2={Northeast}};
		\pic[rotate=15, transform shape] at (120:1) {positioncardfrontdisplay2={Southwest}};
		\pic[rotate=45, transform shape] at (180:1) {positioncardfrontdisplay2={Southeast}};
	\end{tikzpicture}
	\qquad\begin{tikzpicture}[scale=0.1, transform shape, x=1in, y=1in]
		\pic[rotate=-45, transform shape] at (0:1) {cardbackdisplay2={Blue}};
		\pic[rotate=-15, transform shape] at (60:1) {cardbackdisplay2={Blue}};
		\pic[rotate=15, transform shape] at (120:1) {cardbackdisplay2={Blue}};
		\pic[rotate=45, transform shape] at (180:1) {cardbackdisplay2={Blue}};
	\end{tikzpicture}
\end{center}
\end{description}
\newpage
\section*{Set Up}
\begin{enumerate}[leftmargin=*, itemindent=0pt]
	\item Arrange any four of the tokens in a square in the play area. This is the \emph{quatrefoil}. The fifth token is the \emph{free token} and should be placed nearby.
	\item Deal one colour card and one position card face down to each player. You may look at your own cards, but not those dealt to other players.
	\item Place the remaining colour card face down in the play area to one side of the quatrefoil. This card indicates which direction is \emph{north}.
\end{enumerate}
\vfill
\begin{center}
\begin{tikzpicture}[scale=0.125, x=1in, y=1in, transform shape]
\pic () at (-2.25,1) {token2={Blue}};
\pic () at (-0.25,1) {token2={Green}};
\pic () at (-2.25,-1) {token2={Purple}};
\pic () at (-0.25,-1) {token2={Red}};
\pic () at (2,0) {token2={Yellow}};

\pic () at (-1.75,8.75) {cardbackdisplay2={Blue}};
\pic () at (1.75,8.75) {cardbackdisplay2={Red}};
\path (0,0) to (0,12);

\pic[rotate=90] at (6.25,1.75) {cardbackdisplay2={Blue}};
\pic[rotate=90] at (6.25,-1.75) {cardbackdisplay2={Red}};

\pic () at (-1.75,-5.25) {cardbackdisplay2={Red}};
\pic () at (1.75,-5.25) {cardbackdisplay2={Blue}};

\pic[rotate=90] at (-6.5,1.75) {cardbackdisplay2={Red}};
\pic[rotate=90] at (-6.5,-1.75) {cardbackdisplay2={Blue}};

\pic[rotate=90] at (0,4.75) {cardbackdisplay2={Red}};
\end{tikzpicture}%
\captionof*{figure}{\emph{The play area after set up is complete.}}
\end{center}%
\newpage
\section*{Player Turns}
On your turn, take one of three possible \emph{actions}:
 \begin{description}[leftmargin=0pt, labelsep=\widthof{\ }]
	\item[Push \textendash] Place the free token next to any row or column of the quatrefoil. Slide that entire row or column by pushing on the free token until it is part of the quatrefoil.
	\vfill
	\begin{center}
	\begin{tikzpicture}[scale=0.125, x=1in, y=1in, transform shape]
\pic () at (0,0) {token2={Yellow}};
\pic () at (2,0) {token2={Blue}};
\pic () at (4,0) {token2={Green}};
\pic () at (2,-2) {token2={Purple}};
\pic () at (4,-2) {token2={Red}};

\pic () at (7,-1) {nextarrow};

\pic () at (10,0) {token2={Yellow}};
\pic () at (12,0) {token2={Blue}};
\pic () at (14,0) {token2={Green}};
\pic () at (11,-2) {token2={Purple}};
\pic () at (13,-2) {token2={Red}};

\pic () at (17,-1) {nextarrow};

\pic () at (20,0) {token2={Yellow}};
\pic () at (22,0) {token2={Blue}};
\pic () at (24,0) {token2={Green}};
\pic () at (20,-2) {token2={Purple}};
\pic () at (22,-2) {token2={Red}};
	\end{tikzpicture}
	\captionof*{figure}{\emph{Pushing the free token into a row.}}
	\end{center}
	\vfill
		\begin{center}
	\begin{tikzpicture}[scale=0.125, x=1in, y=1in, transform shape]
\pic () at (3,2) {token2={Yellow}};
\pic () at (1,0) {token2={Blue}};
\pic () at (3,0) {token2={Green}};
\pic () at (1,-2) {token2={Purple}};
\pic () at (3,-2) {token2={Red}};

\pic () at (7,-1) {nextarrow};

\pic () at (13,1) {token2={Yellow}};
\pic () at (11,0) {token2={Blue}};
\pic () at (13,-1) {token2={Green}};
\pic () at (11,-2) {token2={Purple}};
\pic () at (13,-3) {token2={Red}};

\pic () at (17,-1) {nextarrow};

\pic () at (21,0) {token2={Blue}};
\pic () at (23,0) {token2={Yellow}};
\pic () at (21,-2) {token2={Purple}};
\pic () at (23,-2) {token2={Green}};
\pic () at (23,-4) {token2={Red}};
	\end{tikzpicture}
	\captionof*{figure}{\emph{Pushing the free token into a column.}}
	\end{center}
	\vfill
 	Notice that this will also eject a token from the quatrefoil. The ejected token becomes the free token for the next player's turn.
 	\newpage
	\item[Twist \textendash] Rotate the quatrefoil by ninety degrees in either direction (clockwise or anticlockwise).
	\vfill
	\begin{center}
	\begin{tikzpicture}[scale=0.125, x=1in, y=1in, transform shape]
	\pic () at (1,0) {token2={Blue}};
	\pic () at (3,0) {token2={Green}};
	\pic () at (1,-2) {token2={Purple}};
	\pic () at (3,-2) {token2={Red}};

	\pic () at (7,-1) {nextarrow};

	\pic[rotate around={-45:(1,-1)}] () at (11,0) {token2={Blue}};
	\pic[rotate around={-45:(-1,-1)}] () at (13,0) {token2={Green}};
	\pic[rotate around={-45:(1,1)}] () at (11,-2) {token2={Purple}};
	\pic[rotate around={-45:(-1,1)}] () at (13,-2) {token2={Red}};

	\pic () at (17,-1) {nextarrow};

	\pic[rotate around={-90:(1,-1)}] () at (21,0) {token2={Blue}};
	\pic[rotate around={-90:(-1,-1)}] () at (23,0) {token2={Green}};
	\pic[rotate around={-90:(1,1)}] () at (21,-2) {token2={Purple}};
	\pic[rotate around={-90:(-1,1)}] () at (23,-2) {token2={Red}};
	\end{tikzpicture}
	\captionof*{figure}{\emph{Twisting the quatrefoil clockwise.}}
	\end{center}
	\vfill
	\begin{center}
	\begin{tikzpicture}[scale=0.125, x=1in, y=1in, transform shape]
	\pic () at (1,0) {token2={Blue}};
	\pic () at (3,0) {token2={Green}};
	\pic () at (1,-2) {token2={Purple}};
	\pic () at (3,-2) {token2={Red}};

	\pic () at (7,-1) {nextarrow};

	\pic[rotate around={45:(1,-1)}] () at (11,0) {token2={Blue}};
	\pic[rotate around={45:(-1,-1)}] () at (13,0) {token2={Green}};
	\pic[rotate around={45:(1,1)}] () at (11,-2) {token2={Purple}};
	\pic[rotate around={45:(-1,1)}] () at (13,-2) {token2={Red}};

	\pic () at (17,-1) {nextarrow};

	\pic[rotate around={90:(1,-1)}] () at (21,0) {token2={Blue}};
	\pic[rotate around={90:(-1,-1)}] () at (23,0) {token2={Green}};
	\pic[rotate around={90:(1,1)}] () at (21,-2) {token2={Purple}};
	\pic[rotate around={90:(-1,1)}] () at (23,-2) {token2={Red}};
	\end{tikzpicture}
	\captionof*{figure}{\emph{Twisting the quatrefoil anticlockwise.}}
	\end{center}
\vfill
	\item[Stack \textendash] Place the free token on top of the quatrefoil. The game ends immediately after any player takes this action.
	\vfill
	\begin{center}
	\begin{tikzpicture}[scale=0.125, x=1in, y=1in, transform shape]
	\pic () at (1,0) {token2={Blue}};
	\pic () at (3,0) {token2={Green}};
	\pic () at (1,-2) {token2={Purple}};
	\pic () at (3,-2) {token2={Red}};
	\pic () at (-3,-1) {token2={Yellow}};

	\pic () at (7,-1) {nextarrow};

	\pic () at (11,0) {token2={Blue}};
	\pic () at (13,0) {token2={Green}};
	\pic () at (11,-2) {token2={Purple}};
	\pic () at (13,-2) {token2={Red}};
	\pic () at (12,-1) {token2={Yellow}};
	\end{tikzpicture}
	\captionof*{figure}{\emph{Stacking the free token on the quatrefoil.}}
	\end{center}
\end{description}
\newpage
\section*{Turn Order}
Play proceeds to the left (clockwise). That is, after you finish your turn the player to your left will go next.

\section*{Correct Positions}
The colour cards and the position cards together specify the correct position for each token.

The correct positions of four of the tokens are specified by the cards held by the players. These tokens should be in the quatrefoil. Your position card specifies the correct position in the quatrefoil for the token specified by your colour card.

The remaining token should be the free token. Because it should not be in the quatrefoil, it does not have an associated position card.

\section*{Communication}
During the game, communication between players is strictly limited. At any time, you may only indicate whether your token is in the correct position or not.

This should be done using a simple nonverbal signal. Many players like to give a ``thumbs up'' gesture when their token is in the correct position.
\newpage
\section*{Winning the Game}
The game ends when anyone takes the stack action.

Everyone wins if all of the tokens are in their correct positions when the game ends. If any token is not in its correct position, then everyone loses.

\section*{Variants}
There are several ways to modify the rules of the game to adjust its difficulty:
\begin{description}[leftmargin=0pt, labelsep=\widthof{\ }]
	\item[Nothing to Hide -] Turn all of the colour cards and position cards face up. Try to solve the puzzle in as few moves as possible. This is a good variant to try when you are first learning how to play.
	\item[Half and Half -] Pick either colour cards or position cards. Turn all of those cards face up. Keep the other cards face down. Try to solve the puzzle in as few moves as possible. This is a good variant to try when playing with all cards face up gets too easy.
	\item[Race the Clock -] As in the base game, deal all of the colour cards and position cards face down. After you look at your cards, start a two-minute timer. Try to solve the puzzle before time expires. This is a good variant to try when you have mastered the basic game and want a tougher challenge.
\end{description}
\newpage
\section*{Quatrefoil Dual}
Another way to play is to reveal the solution to the puzzle up front, but limit what actions each player can take.

To play this way, set up the tokens in the middle of the play area, deal one position card to each player, and place one colour card near the quatrefoil as normal.

Then, deal the remaining colour cards face up in a square near the quatrefoil. These four cards specify the correct position for each token.

During the game, anyone can take an action at any time. You may take the twist or stack actions as normal. You may only take a push action that concludes with the free token in the position indicated on your position card.

You may only communicate via your actions.

The game ends when anyone takes the stack action. Everyone wins if all of the tokens are in their correct positions. Otherwise, everyone loses.

For an additional challenge, start a two-minute timer immediately after dealing the last card. Try to solve the puzzle before time expires.
\vfill
\textbf{Contact}: \href{mailto:quatrefoil.game@gmail.com}{quatrefoil.game@gmail.com}

\begin{tabular}{@{}m{\columnwidth-\widthof{\Huge{\doclicenseIcon}}-0.5cm}@{\hspace{0.05cm}}m{\widthof{\Huge{\doclicenseIcon}}}@{}}
{This work is licensed under a\newline ``CC BY 4.0'' license.} & \Huge{\doclicenseIcon}\\
\end{tabular}


\end{document}
